\documentclass[12pt,a4paper,sans]{moderncv}
\moderncvstyle{classic}
\moderncvcolor{black}
\usepackage[scale=0.9]{geometry}
\setlength{\hintscolumnwidth}{100pt}
\usepackage{multicol}
\usepackage[utf8]{inputenc}
\usepackage[russianb]{babel}
\usepackage{fontawesome}
\firstname{Антон}
\familyname{Сивков}
\mobile{+7 (926) 049-10-75}
\newcommand{\myurl}[1]{\color{blue}\url{#1}\color{black}}
\email{anton6722@gmail.com}
\social[github]{antosiv}
\photo[80pt][0.4pt]{ava}

\begin{document}
\makecvtitle

\section{Опыт Работы}
\cventry{Июнь 2018 -- now}{BostonGene (\myurl{https://bostongene.com})}{Аналитик данных research \& development отдела}{}{}{
	\begin{itemize}
		\item В основном занимаюсь разработкой софта для научного отдела на python, участвую в разработке пайплайна обработки Next Generation Sequencing данных и побочных web-сервисов, активно работаю с ETL/web фреймворками и СУБД
		\item Также занимаюсь Natural language processing, в основном простыми задачами, связаными с обработкой медицинских текстов
	\end{itemize}
}

\section{Образование}
\cventry{2015 -- 2019}{Высшее}{Факультет Вычислительной Математики и Кибернетики МГУ имени М. В. Ломоносова}{бакалавр}{\textit{cредний балл -- 4.45 / 5}}{}{}{}
\cventry{2019 -- now}{Неоконченное высшее}{НИУ ВШЭ, Магистерская программа «Анализ данных в биологии и медицине»}{}{}{}{}{}
\cventry{2018}{Техносфера@mail.ru, 2 семестра, неокончена}{}{}{}{}


\section{Навыки}

\cvitem{Языки \newline программирования}{
    \begin{itemize}
        \item Продвинутый уровень: Python3.5-3.8
        \item Базовый уровень: C++, SQL, Python2.7
    \end{itemize}
}
\cvitem{Инструменты web-разработки}{nginx, flask}
\cvitem{ETL - фреймворки}{Apache Airflow + Celery, Prefect \myurl{https://github.com/PrefectHQ/prefect}{}}
\cventry{Инструменты ML}{Pandas, NumPy, SciPy, scikit-learn, matplotlib, pytorch, keras}{}{}{}{}
\cvitem{Другие инструменты}{
    \begin{itemize}
        \item Продвинутый уровень: git, docker, pip, sqlalchemy
        \item Базовый уровень: PostgreSQL, MongoDB, vim, virtualenv, linux
    \end{itemize}
}
\cvitem{Другое}{Алгоритмы и структуры данных, ООП, Atlassian JIRA}

\cvitem{Языки}{Русский, Английский (intermediate)}

\pagebreak

\section{Проекты}
\cventry{2018}{Реализация 5 моделей машинного обучения}{}{\myurl{https://github.com/antosiv/study_progs/tree/master/python/ML_pack}}{}{
	\begin{itemize}
		\item KNeighborsClassifier
		\item DecisionTreeClassifier
		\item SGDClassifier, svm learning with sgd
		\item Kmeans, clastering
		\item EMGaussianMixture expectation maximization Gaussian mixture inference
	\end{itemize}
	Модели реализовывались как домашнее задание курса Техносферы по машинному обучению, собрал все в один пакет
}

\cventry{2018}{Анализ тональности отзывов отелей}{}{\myurl{https://github.com/antosiv/dm1project}}{}{
    \begin{itemize}
        \item Проект выполнялся как выпускное задание курса Техносферы по машинному обучению
        \item Данные собирались вручную, был произведен парсинг отзывов на самый большой отель в мире с сайта \myurl{https://booking.com}
        \item Проект выполнялся вв команде, я занимался парсингом и TF-IDF, было построено несколько классификаторов на данных TF-IDF и word2vec, лучший показал accuracy 0.9
    \end{itemize}
}

\cventry{2019}{Диплом, моделирование простого объекта управления с помощью различных нейронных сетей}{}{\myurl{https://github.com/antosiv/study_progs/tree/master/jupyter/diploma}}{}{
	\begin{itemize}
		\item Эксперименты реализованы на языке python с помощью инструментов pytorch и control
		\item Проведены эксперименты с многослойным перцептроном, LSTM, обучающейся по входным данным объекта, LSTM, обучающейся по входным и выходным данным
		\item В результате LSTM сети качественно промоделировали объект, показали способность к экстраполяции его поведения
	\end{itemize}
}

\section{О себе}
\cvitem{Цели}{Я хочу стать профессиональным разработчиком, работать над интересными пректами и писать хороший код; мне нравится изучать новые технологии и практики программирования, я часто стараюсь разобраться в сути задач, которые решаю.}
\cvitem{Дополнительные курсы}{
    \begin{itemize}
        \item Курс ''Погружение в Python'', \myurl{https://www.coursera.org/account/accomplishments/certificate/YSM43L2WE49G}
        \item Курс ''Algoritmic Toolbox'', \myurl{https://www.coursera.org/account/accomplishments/certificate/AALYRUSNGRTR}
        \item Курс ''Теория Вероятностей'', \myurl{https://stepik.org/cert/215037}
    \end{itemize}
}
\cvitem{Дополнительно}{Хочу добавить, что способен быстро разбираться в новых проектах и фреймворках, люблю исследовать и править сложные баги.}

\end{document} 
